\documentclass[a4paper]{article}
\usepackage{graphicx}
\usepackage{onecolpceurws}
\usepackage[utf8]{inputenc}

\title{Granularity Social Network Analysis: Research in Progress}

\author{
Miguel Ángel Fernández \\ GSyC/LibreSoft \\
                Universidad Rey Juan Carlos \\ ma.fernandezsa@alumnos.urjc.es
\and
Gregorio Robles \\ GSyC/LibreSoft \\
                Universidad Rey Juan Carlos \\ grex@gsyc.urjc.es
\and
Jesús M. González Barahona \\ Bitergia \\
                jgb@bitergia.com
}

\institution{}


\begin{document}
\maketitle

\begin{abstract}
The Abstract paragraph should be indented $1.25$ inch on both
the left and right-hand margins. Abstract must be centered, bold, and
in point size 12. Two line spaces precede the Abstract. The Abstract
must be limited to one paragraph.
\end{abstract}
\vskip 32pt


\section{Introduction/Motivation}

Most social network studies for software projects are focused in a file-based data of interactions in that network of developers. If there is a collaboration between two or more developers in same file, how can we be sure when there are thousands lines of code if that people did actually collaborate? \\
We want to go to a deeper level: in this particular case, a method-base analysis which allows us to know with more accuracy if there is a real relationship between those developers.

\section{Methodology}

The program studies registered changes made in a given repository tracked by Git. From that repository, it extracts log from a specified date range. Using that data, the program iterates with each commit made, and does a checkout with all of them to take the repository to the state the program was when each commit was made. \\ At each iteration, we use Ctags with each file to get classified all changes made in each of those files. Next, is to find matches between the commit info and Ctags data, so we can tell if a method has been modified.\\
Once the program finishes all checkout iterations, it takes that data and searches for coincidences when different developers have modified same file and method and outputs that data into a CSV-format file, which can be used in programs like Gephi to watch a graph-type representation of the network.

\subsection{Particular case: GNOME gedit}

This is a subsection

\subsubsection{Figures}

Figure \ref{fig1} shows how to include a figure as encapsulated postscript.
The source of the figure is in file {\tt fig1.eps}.

\begin{figure}[ht]
\begin{center}
\includegraphics[height=4cm]{fig1}
\caption{Sample EPS figure }
\label{fig1}
\end{center}
\end{figure}

Below is another figure using LaTeX commands.


\begin{figure}[ht]
\begin{center}
\setlength{\unitlength}{1pt}
\footnotesize
\begin{picture}(160,80)
        \put(0,0){\framebox(160,80)[]{}}
        \put(10,35){\framebox(80,40){}}
        \put(100,20){\framebox(40,20){}}
        \put(70,10){\framebox(20,10){}}
        \put(20,5){\framebox(10,5){}}
\end{picture}
\caption{Sample Figure Caption}
\end{center}
\end{figure}

\subsubsection{Tables}

All tables must be centered, neat, clean and legible. Do not use pencil
or hand-drawn tables. Table number and title always appear before the
table.

One line space before the table title, one line space after the table
title and one line space after the table. The table title must be
initial caps and each table numbered consecutively.

\begin{table}[ht]
\begin{center}
\caption{Sample Table}

\bigskip

\begin{tabular}{|l|l|r|}
\hline
A & B & 1\\ \hline
C & D & 2\\
E & F & 3\\ \hline
\end{tabular}
\end{center}
\end{table}


\subsubsection{Handling References}

Use a first level heading for the references. References follow the
acknowledgements.


\subsubsection{Acknowledgements}

Use a third level heading for the acknowledgements. All acknowledgements
go at the end of the paper.


\bibliographystyle{alpha} 
\bibliography{bib}
%inline the .bbl file directly for mailing to authors.

%\begin{thebibliography}{Com79}
%
%\bibitem[Com79]{Comer-btree}
%D.~Comer.
%\newblock The ubiquitous b-tree.
%\newblock {\em Computing Surveys}, 11(2):121--137, June 1979.
%
%\bibitem[Knu73]{Knuth-vol3}
%D.~E. Knuth.
%\newblock {\em The Art of Computer Programming -- Volume 3 / Sorting and
%  Searching}.
%\newblock Addison-Wesley, 1973.
%
%\end{thebibliography}

\end{document}


