\documentclass[a4paper]{article}
\usepackage{graphicx}
\usepackage{url}
\usepackage{onecolpceurws}
\usepackage[utf8]{inputenc}

\title{Collaboration Networks in Software Development: Perspectives from Applying different Granularity Levels using Social Network Analysis - Research in Progress}

\author{
Miguel Ángel Fernández, Gregorio Robles \\ GSyC/LibreSoft \\
                Universidad Rey Juan Carlos \\ \{ma.fernandezsa@alumnos., gregorio.robles@\}urjc.es
\and
Jesús M. González Barahona \\ Bitergia \\
                jgb@bitergia.com
}

\institution{}


\begin{document}
\maketitle

\begin{abstract}
This paper shows research in progress in the analysis of collaboration networks
found in software development projects. Traditionally, collaboration networks
are obtained by analyzing collaboration in the same file or module/directory;
when two developers perform modifications on the same entity during a given time
period it is assumed that they are at least implictly collaborating. In our 
research, we want to study how the granularity of the software artifact affects
the research output of collaboration graphs. In this regard, we obtain traditional
graphs based on collaboration in files and augment it with information of
collaboration at the function/method level. In the future we want to include
deveoper affiliation information to perform a collaboration analysis at the
company level.
\end{abstract}
\vskip 32pt


\section{Introduction/Motivation}

The development of large software systems is a collaborative task where
many developers, sometimes up to thousands of them, are involved. In such 
scenario, software engineering research has been long looking after
understanding how these collaborations arise, and how they evolve over time~\cite{minto2007recommending,singh2010small,surian2010mining,hossain2009social}.

In order to identify collaboration, many scholars have used techniques such as
social network analysis, where two developers (nodes) are connected if they
have collaborated together~\cite{madey2002open,lopez2004applying}. In most social network studies the
resulting network is based on file-based or module-based data of interactions;
if there has been a collaboration between two developers in a file or a module,
these developers are connected. 

Our research is concerned with the fact that the resulting network graph depends
heavily on the granularity level that is selected~\cite{howison2012validity}. When there are tens of files
in a module/directory or thousands of lines in a file, did collaboration really
exist?

Therefore, in addition to the existing collaboration graphs, we have been working
to obtain a new one that takes collaboration at the function/method level into
account. In this type of graph, two developers collaborate if they have modified
the same function in a given time period. We think that, although there might
be exceptions with large fuctions/methods, this provides a new, still unknown
level of granularity in the analysis that can help to obtain a better global 
picture.

\section{Methodology}

Our methodology studies registered changes made to a given repository tracked by
a versioning system (in our case \texttt{git}). 
From that repository, it extracts the log for a specified date range. Using
that data, the program iterates for each commit, performs a checkout
and uses \texttt{ctags} with each file to identify function/method information
in each of those files.

Next, matches between the commits information and \texttt{ctags} data
(still for each checkout) are searched for, to identify those methods that have
changed. By now, changes in methods are only tracked if the method has not 
changed its name. Future versions of our tool will include heuristics to analyze
as well those functions whose name change.

While changes to functions/methods are extracted, the developers who have performed
these chages are attached to the change. As by now, we do not apply developer merging
algorithms, so that a developer with different aliases will only appear once,
but we plan to do it in the future. This data is aggregated and offered
in two CVS-formatted files, one for collaborations occurring in the same file
and another one with collaborations in the same function/method. This output
can be used by traditional programs to obtain a social network grach, such as
Gephi, or to calculate social network measures and properties.

The tools is available on-line as free software in a GitHub repository\footnote{\url{https://github.com/LibreSoftTeam/R-SNA}}.

\section{Case study}

We have used the program to study the evolution of the GNOME-text editor \texttt{gedit}\footnote{The repository of \texttt{gedit} used in this analysis is \url{https://github.com/GNOME/gedit}}.

The considered date range for this study goes from the very beginning of
the project (at least, from the moment a first commit has been found in the log), which is April
15, 1998 until April 15, 2015.

Within this time range, we have chosen time-lapses of six months so
we can handle and understand better the resulting data and its evolution
over time. The chosen time period is consistent with the release period of GNOME 
(at least from 2005 onwards).

In this paper we show the results for two selected timeframes: from January 1, 2001 to May 31, 2001 and from June 1, 2014 to December 31, 2014. For both ranges we offer a graphic representation of collaboration networks between
developers. Each node represents a developer, and edges represents
interactions between them. We have two different graphs for each date range: an in-file collaboration network (developers who have modified same file) and an in-method
collaboration network (developers who have modified same method).

Figure~\ref{fig:2001} shows the different graphs (in-file and in-method data) for the
first half of 2001. As expected the number of collaborations (only those developers
who collaborate are shown in the graph) is larger in the in-file network than in
the in-method one.

\begin{figure}[h!]
\begin{center}
\includegraphics[scale=0.17]{g2001files.png} 
\includegraphics[scale=0.17]{g2001methods.png}
\caption{In-file (left) and In-method (right) collaboration graphs - 2001}
\label{fig:2001}
\end{center}
\end{figure}

Figure~\ref{fig:2014} shows graphs (in-file and in-method data) for the
second half of 2014.

\begin{figure}[h!]
\begin{center}
\includegraphics[scale=0.17]{g2014files.png} 
\includegraphics[scale=0.17]{g2014methods.png}
\caption{In-file (left) and In-method (right) collaboration graphs - 2014}
\label{fig:2014}
\end{center}
\end{figure}

We also have computed a numerical representation, the betweenness centrality of a node,
that reflects the amount of control that
this node exerts over the interactions of other nodes in the network. 
The values of betweenness given in Tables~\ref{tab:2001} and Table~\ref{tab:2014} 
are normalized, and only nodes that have at least one non-zero value are shown.

\begin{table}[ht]
\begin{center}
\caption{Betweenness centrality for first half of 2001}
\label{tab:2001}
\bigskip
\begin{tabular}{|l|l|r|}
\hline
Node & Files & Methods/Functions \\ \hline
Jacob Leach & 0.0015873015873015873 & 0.007368421052631579\\
Jason Leach & 0.0015873015873015873 & 0.06807017543859648\\
Gediminas Paulauskas & 0.08304673721340389 & 0\\
Pablo Saratxaga & 0.10482804232804235 & 0.23964912280701753\\
Paolo Maggi & 0.12272927689594355 & 0.02140350877192982\\
Chema Celorio & 0.190189594356261 & 0.02140350877192982\\ \hline
\end{tabular}
\end{center}
\end{table}

\begin{table}[ht]
\begin{center}
\caption{Betweenness centrality for second half of 2014}
\bigskip
\label{tab:2014}
\begin{tabular}{|l|l|r|}
\hline
Node & Files & Methods/Functions \\ \hline
Igor Gnatenkov & 2.7210884353741496E-4 & 0\\
Alexandre Franke & 6.122448979591836E-4 & 0\\
Matthias Clasen & 0.008321995464852606 & 0.02217741935483871\\
Boris Egorov & 0.006448979591836735 & 0\\
Ignacio Casal Quinteiro & 0.02728312277291869 & 0.0013440860215053762\\
Piotr Drg & 0.027968901846452864 & 0\\
Jesse van den Kieboom & 0.029569160997732425 & 0.01989727342549923\\
Robert Roth & 0.030327502429543254 & 0.0041042626728110595\\
Paolo Borelli & 0.04774570780693228 & 0.012840821812596005\\
Sebastien Lafargue & 0.058611920958859746 & 0.03878648233486943\\
Sebastien Wilmet & 0.14896080336896667 & 0.00842453917050691\\ \hline
\end{tabular}
\end{center}
\end{table}


\section{Future work}

As future work, we plan to:

\begin{itemize}
  \item Include in our tool algorithms to track function name changes~\cite{godfrey2005using}.
  \item Include in our tool algorithms to merge developer aliases~\cite{kouters2012s}.
  \item Augment the analysis with developer affiliation information~\cite{gonzalez2013understanding}. This 
  \item Perform several analysis to compare the collaboration graphs provided by 
  creating collaboration
\end{itemize}

\section*{Acknowledgements}

This work has been funded in part by the Spanish Gov. under SobreSale (TIN2011-28110) and by the Comunidad de Madrid under ``eMadrid - Investigaci\'on y Desarrollo de tecnolog\'ias para el e-learning en la Comunidad de Madrid'' (S2013/ICE-2715).

\bibliographystyle{alpha} 
\bibliography{bib}
%inline the .bbl file directly for mailing to authors.

%\begin{thebibliography}{Com79}
%
%\bibitem[Com79]{Comer-btree}
%D.~Comer.
%\newblock The ubiquitous b-tree.
%\newblock {\em Computing Surveys}, 11(2):121--137, June 1979.
%
%\bibitem[Knu73]{Knuth-vol3}
%D.~E. Knuth.
%\newblock {\em The Art of Computer Programming -- Volume 3 / Sorting and
%  Searching}.
%\newblock Addison-Wesley, 1973.
%
%\end{thebibliography}

\end{document}


